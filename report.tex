% This is a simple LaTex sample document that gives a submission format
%   for IEEE PAMI-TC conference submissions.  Use at your own risk.

% Make two column format for LaTex 2e.
\documentclass[11pt]{article}
\usepackage{times}
\usepackage{picinpar}
\usepackage{epsfig}
\usepackage{scrextend}
%\usepackage{photo}

% Use following instead for LaTex 2.09 (may need some other mods as well).
% \documentstyle[times,twocolumn]{article}

% Set dimensions of columns, gap between columns, and paragraph indent 
\setlength{\textheight}{8.875in}
\setlength{\textwidth}{6.875in}
\setlength{\columnsep}{0.3125in}
\setlength{\topmargin}{0in}
\setlength{\headheight}{0in}
\setlength{\headsep}{0in}
\setlength{\parindent}{1pc}
\setlength{\oddsidemargin}{-.1875in}  % Centers text.
\setlength{\evensidemargin}{-.1875in}

% Add the period after section numbers.  Adjust spacing.
\newcommand{\Section}[1]{\vspace{-8pt}\section{\hskip -1em.~~#1}\vspace{-3pt}} 
\newcommand{\SubSection}[1]{\vspace{-3pt}\subsection{\hskip -1em.~~#1}
     	\vspace{-3pt}}


\begin{document}

% Don't want date printed
\date{}

% Make title bold and 14 pt font (Latex default is non-bold, 16pt) 
\title{\Large\bf Building a Neural Network that Understands Paintings}

\author{Chen Liu \\
 Computer Science Department \\
 Carnegie Mellon University \\
 Pittsburgh, PA 15213}

\maketitle

\section*{\centering Abstract}

\begin{addmargin}[3em]{3em}
This is the abstract of my paper.  It must fit within the 
size allowed, which is about 3 inches, including section 
title, which is 11 point bold font.  If you don't want 
the text in italics, simply remove the `em' command and 
the curly braces which bound the abstract text.  If you 
have em commands within an italicized abstract, the text 
will come out as normal (non-italicized) text. 
\end{addmargin}

\Section{Introduction}

Convolutional neural networks (CNNs) are a type of feed-forward neural network inspired by the biological processes in animals' visual systems. Each CNN has multiple layers of neurons, each looking at a small part of the image called receptive fields, with overlaps between neighboring neurons. CNNs usually consist of convolution layers, pooling layers and fully connected layers. In a convolution layer, filters of small sizes are convolved with the input image of that layer to extract some specific features. The deeper the layer is in, the more abstract and global the features extracted by the convolution layer are.\\
CNNs are shown to be similar to how the neurons are organized in the brain; specifically, they are similar to what Hubel and Wiesel found as simple cells and complex cells in the primary visual cortex

\SubSection{Previous Work}

There are various bibliographic and citation schemes available in
LaTex, but we choose to use the simplest one in this example.
Throughout I may cite references of the form \cite{key:foo} or
\cite{foo:baz}, and LaTeX will keep track of numbering.  The numbers
are based on the order you place them in the bibliography, not the
order they appear in the text.  They should (I believe) be in
alphabetical order.  LaTex will put square brackets about the number
within the text of your paper.  For those of you new to LaTex, you may
have to run the latex process twice to allow all references to be
resolved. You will get a warning about a missing .aux file.  Just
rerun latex and it will be ok.

\Section{Summary and Conclusions}

This template will get you through the minimum article, i.e., with no
figures or equations.  To include those, please refer to your LaTeX
manual and the IEEE publications guidelines.  However, for a vision
conference you will probably want the following equation somewhere:
$$g(x) = {1\over\sqrt{2\pi}\sigma}e^{-x^2/2\sigma^2}$$
Good Luck!

% This is how to do an unnumbered section (note asterisk).
\section*{Acknowledgments}

This is how to do an unnumbered subsection.  For submission, there
should be no acknowledgments as this could lead to identification
of the author.

\begin{thebibliography}{9}
\small  % Use 9 point text.

\bibitem{key:foo}
I. M. Author,
``Some Related Article I Wrote,''
{\em Some Fine Journal}, Vol. 17, pp. 1-100, 1987.
\bibitem{foo:baz}

A. N. Expert,
{\em A Book He Wrote,}
His Publisher, 1989.

\end{thebibliography}

\end{document}